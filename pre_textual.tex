
\maketitle
\makecover % folha de rosto               (obrigatório)
% Se houver errata, deverá ser incluída após a folha de rosto


% ---
% Após a apresentação do trabalho incluir o pdf com a folha com as assinaturas utilizando os seguintes comandos.
% 
% \begin{titlingpage*}
% \includepdf{folha_assinada.pdf}
% \end{titlingpage*}
%
\makeapproval %  (obrigatório)

\dedication{Folha na qual o autor presta uma homenagem ou dedica seu trabalho. Não leva título e a dedicatória deve aparecer na parte inferior da página, a 8cm da margem esquerda.}

\clearpage\pagestyle{empty}
\chapter*{Agradecimentos}
"Expressos pelo autor que presta seu reconhecimento às pessoas e instituições que
colaboraram na elaboração do seu trabalho. Não leva indicativo numérico e o título
deve ser centralizado na folha, utilizando a mesma tipologia das seções primárias do
texto." conforme o manual de normalização página 10.
\clearpage

% \input{1-Pre-Textual/Epigrafe.tex}%       (opcional)

\epigraph{only a boring man will always want things to match;
real quality lies in irregularity}{Yoshida Kenkō}

\clearpage\pagestyle{empty}
% \input{1-Pre-Textual/Resumo.tex} %        (obrigatório)

\begin{abstract}
    É a apresentação concisa dos pontos relevantes do texto, fornecendo uma visão rápida
    e clara do conteúdo do trabalho e das conclusões alcançadas, de tal forma que este
    possa dispensar a consulta ao original. Deve ter uma extensão de 150 a 500 palavras.
    Abaixo do resumo devem figurar as palavras-chave, representativas do conteúdo do
    trabalho. Devem ser precedidas da expressão Palavras-chave: separadas entre si por
    ponto e finalizadas também por ponto.
    \\ \\
    \textbf{Palavras-chave:} Palavra-chave 1; Palavra-chave 2; Palavra-chave 3; Palavra-chave 4; Palavra-chave 5.
\end{abstract}
\clearpage

% \input{1-Pre-Textual/Abstract.tex} % 
\begin{otherlanguage}{brazil}
    \begin{abstract}
        Tradução do resumo em língua vernácula preferencialmente para o inglês. Deve ser
        seguido das palavras-chave traduzidas para a mesma língua.
        Localizado logo após o resumo em língua vernácula.
        \\ \\
        \textbf{Keywords:} Keywords 1; Keywords 2; Keywords 3; Keywords 4; Keywords 5.
    \end{abstract}
\end{otherlanguage}
\clearpage


\listoffigures* %                          (opcional)
% \listoftables*  %                          (opcional)


% --- Lista de abreviaturas (opcional)
% Lista de abreviaturas usando o pacote acronym. As abreviaturas precisam ser usadas no texto usando o comando \ac{} na primeira vez que for invocado ele substituirá a descrição longa e nas vezes subsequentes incluirá apenas a sigla com um link para a definição na lista de abreviaturas 
% ---

% \chapter*{Lista de Abreviaturas e Siglas}
% \begin{acronym}
%     \acro{API}{Interface de Programação de Aplicativos, do inglês \textit{Application Programming Interface}}
%     \acro{CPU}{Unidade Central de Processamento, do inglês \textit{Central Processing Unit}}
%     \acro{DAG}{Grafos Acíclicos Dirigidos, do inglês \textit{Directed Acyclic Graph}}
%     \acro{GC}{Coletor de Lixo, do inglês \textit{Garbage Collector}}
%     \acro{VM}{Maquina Virtual, do inglês \textit{Virtual Machine}}
% \end{acronym}


% ---
% inserir lista de símbolos
% ---
\nomenclature{$\alpha$}{Letra grega minúscula alpha}
\nomenclature{$\rightarrow$}{"Tem seu valor alterado para..."}
\nomenclature{$\propto$}{"é proporcional à..."}
\nomenclature{$\lesssim$}{"Aproximadamente menor que..."}
\nomenclature{$\gg$}{"Muito maior que..."}
\printnomenclature

\clearpage

\tableofcontents* %                        (obrigatório)